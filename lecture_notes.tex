\documentclass[10pt,a4paper]{article}
\usepackage[utf8]{inputenc}
\usepackage[letterpaper, margin=1in]{geometry}
\usepackage{amsmath}
\usepackage{amsfonts}
\usepackage{amsthm}
\usepackage{amssymb}
\usepackage{titlesec}
\usepackage{mathtools}
\usepackage{enumerate}
\usepackage[T1]{fontenc}

\theoremstyle{definition}
\newtheorem{definition}{Definition}
\theoremstyle{cor}
\newtheorem{cor}{Corollary}
\theoremstyle{theorem}
\newtheorem{theorem}{Theorem}
\theoremstyle{lemma}
\newtheorem{lemma}{Lemma}
\theoremstyle{example}
\newtheorem{example}{Example}

\newcommand{\norm}[1]{\Vert #1 \Vert}

\titleformat{\section}[block]{\Large\bfseries\filcenter}{}{1em}{}
\titleformat{\subsection}[hang]{\bfseries}{}{1em}{}
\author{Stephen Diadamo}
\title{Functional Analysis}
\begin{document}

\section{Functional Analysis Lecture Notes}
\subsection*{October 13, 2015}
\textbf{Function analysis }
\begin{itemize}
	\item is the foundation of modern analysis
	\item grew up in the 1920's
	\item is the mathematical framework for quantum theory and many continuous models in engineering, economics, etc.
	\item combines analysis, topology, and algebra
	\item studies topological vector spaces and mappings between them
	\item generalizes linear algebra to infinite dimensions 
\end{itemize}
\section{Topological Spaces}

\begin{definition}
Let X be a set and $\tau$ a collection of subsets of X. (X, $\tau$) is a $\textit{topological space}$ iff
\begin{enumerate}
\item \{X, $\varnothing \} \in \tau$
\item {U, V} $\in\tau \rightarrow$ U $\cap $ V $\in\tau$
\item $\tau$ is closed under arbitrary unions, that is, if $\forall i\in I : U_i \in\tau$ then $\bigcup_{i \in I} U_i \in \tau$  
\end{enumerate}
\end{definition}
$\tau$ is then called a \textit{topology}, its elements are \textit{open sets} and their complements are \textit{closed sets}.

\begin{definition}
\textbf{Open Set} \\
A set is called open if it is a neighbourhood of every point. 
\end{definition}
\begin{definition}
\textbf{Neighbourhood} \\
A neighbourhood of set $S$ is a set $P$ that contains an open set $U$ so $S \subset U \subset P$. 
\end{definition}

\noindent Topological spaces are the most general framework for 'doing analysis' (i.e. where notions like convergence, continuity, compactness, etc. are defined).  The family of all topologies on $X$ is partially ordered by inclusions if $\tau_1, \tau_2$ are two topologies, we call $\tau_1$ "weaker"/"courser" than $\tau_2$ iff $\tau_1 \subseteq \tau_2$. The "discrete topology" $\tau_{discrete} \coloneqq  \{U \subseteq X\}$ is thus the strongest and the "trivial (or indiscrete) topology" $\tau_{trivial}\coloneqq \{\varnothing, X\}$ the weakest.

\begin{definition}
Let $(X, \tau)$ be a topological space and $A \subseteq X$.
\begin{itemize}
\item The \textbf{closure} of $A$, $\overline{A}$, is the smallest closed set including $A$. (i.e. the intersection of all closed sets containing $A$).
\item The \textbf{interior} of $A$, $\text{int}(A)$, is the largest open set contained in $A$.
\item The \textbf{boundary} of $A$ is defined as $\partial A \coloneqq \overline{A} \setminus \text{int}(A) =  \overline{A} \cap \overline{[X\\A]}$.
\item $A$ is called \textbf{dense} in $X$ iff $\overline{A} = X$.
\item $A$ is called \textbf{neighbourhood} of $x\in X$ iff $\exists U \in \tau : x\in U \subseteq A$.
\item $B \subseteq \tau$ is called a \textbf{base} for $\tau$ iff $\exists U \in \tau$ that is of the form $U = \cup_{\alpha} V_{\alpha}$ for some family $\{ V_{\alpha} \subseteq B$.
\item $(X, tau)$ is called \textbf{Hausdorff Space} iff $\forall x, y \in X, \exists U_x, U_y \in \tau : x\in U_x \wedge y\in U_y \wedge (X \neq y \Rightarrow U_x \cap U_y = \varnothing$.
\item $(X, \tau)$ is called \textbf{seperable} iff it contains a countable, dense, subset $C \subseteq X = \tau$.
\end{itemize}
\end{definition}

\noindent There are two important ways of constructing topologies:
\begin{enumerate}
\item \textbf{Metric Topologies}: If $(X, d)$ is a metric space, then $d$ induces a topology $\tau$ via 
\begin{align*}
U\in \tau \Leftrightarrow \forall x \in U \exists \epsilon > 0 : B_{\epsilon}(x) \subset U
\end{align*}
where $B_{\epsilon} \coloneqq \{y\in X | d(x, y) < \epsilon\}$.
\begin{itemize}
\item $\{B_{\epsilon}(x)\}_{\epsilon > 0, x \in X}$ is then a base for the topology.
\item A topological space for which there is such a metric is called \textbf{metrizable}.
\item Metrizable spaces are Hansdorff. Hence, non-Hansdorff topologies (such as the trival topology if $|X| \geq 2$ or the Zariski topology from algebraic geometry) are not metrizable. 
\item Most notions used in metric  spaces have a topological generalization (see above). Important exceptions are \textbf{completeness}, \textbf{boundedness}, and \textbf{uniform continuity}.
\end{itemize}
\item \textbf{Weak topologies}. We need the notion of continuity for this.
\begin{definition}
Let $(X_i, \tau_i)_{(i \in \{1, 2\})}$ be topological spaces and $f : X_1 \rightarrow X_2$. $f$ is called \textbf{continuous} iff the pre-image of any open set is open. It is called \textbf{open} iff the image of any open set is open and a \textbf{homeomorphism} iff it is an open, continuous bijection. Equivalently iff it is bijective and $f$ as well as $f^{-1}$ are continuous.
\end{definition}
\end{enumerate}
Note that for any function $f:X_1 \rightarrow X_2$ between the sets, there are always topologies with respect to which $f$ is continuous. E.g. if $tau_1$ is the discrete topology or if $\tau_2$ is the trivial topology.
\\ \\
A so-called \textbf{weak topology} is now defined by requiring a family $\overline{f}$ of functions from a set $S$ into a topological space $(X, \tau)$ to be continuous. In order to make this construction, unique, one takes the weakest topology on $S$ for which this is the case.
\\\\
A base for this topology is given by all finite intersections of sets of the form $f^{-1}(u)$ where $f\in\overline{f}$ and $u\in\tau$.
\\\\
Three ways of constructing new topological spaces from old ones:
\begin{enumerate}
\item The \textbf{subspace} topology if a subset $S$ of a topological space $(X, \tau)$ is $\tau_S \coloneqq \{ V\subseteq S | \exists U \in \tau : U \cap S = V\}$. (Elements  of $\tau_S$ are sometimes called \textbf{relatively open}.
\item The \textbf{product} topology of $(X_1, \tau_1)$ and $(X_2, \tau_2)$ is defined as
\begin{align*}
 \{ U \subseteq X_1 \times X_2 | \forall(x, y)\in U \text{ there are open neighbourhoods } U_x \in \tau_1, U_y \in \tau_2 : U_x \times U_y \subseteq U \}
\end{align*}
\item the \textbf{quotient} topology of a quotient $X \setminus \sim$ of $(X, \tau)$ is defined as $\{ U \subseteq X \setminus \sim  | q^{-1}(U) \in \tau \}$ where $q:X \rightarrow X\setminus \sim : q(x) = q(y) \Leftrightarrow x \sim y$.
\end{enumerate}
\begin{definition}
Let $\mathbb{K}\in\{\mathbb{R}, \mathbb{C}\}$ be equipped with the usual (i.e. metric) topology. A $\mathbb{K}$-vector space $X$ together with a topology on $X$ is called \textbf{topological vector space} iff $\cdot : \mathbb{K} \times X \rightarrow X$ and $+ : X\times X \rightarrow X$ are continuous with respect to the product topology. 
\end{definition}

\noindent We will later see that all Banach and Hilbert spaces are examples of topological vector spaces.

\begin{definition}
Let $(X,\tau)$ be a topological space.
\begin{itemize}
\item A sequence $(X_n)$ in $X$ is said to \textbf{converge} to $x_0\in X$ iff for any neighbourhood $U$ of $x_0$ there is an $m\in \mathbb{N}$ such that  $\forall n > m : x_n \in U$.
\item $A \subseteq X$ is called \textbf{sequentially compact} iff every sequence in $A$ has a subsequence converging to an element in $A$.
\item $A \subseteq X $ is called \textbf{compact} iff every open cover of $A$ contains a finite subcover. That is, 
\begin{align*}
\left(A \subseteq \cup_{i \in I} U_i \wedge \{U_i\}_{i\in I} \subseteq \tau \Rightarrow \exists \zeta \subseteq I : |\zeta| < \infty \wedge \cup_{i\in \zeta} U_{i} \supseteq A\right)
\end{align*}
\end{itemize}
\end{definition}
\noindent There is no general relation between compactness and sequential compactness. In metric spaces, however, they are equivalent. 

\begin{cor}
If $A$ is a closed subset of a compact topological space $(X, \tau)$, then $A$ is compact.
\end{cor}
\begin{proof}
For any open over $\cup_{i\in I} U_i \supseteq A, (\cup_{i\in I} U_i)\cup (X\setminus A) = X$ is an open cover of $X$ with finite subcover $\cup_{i\in\zeta} U_{i} \cup (X \setminus A) = X$. Hence, $\cup_{i \in \zeta} U_i \supseteq A$.
\end{proof}

\begin{cor}
A compact set $K$ in a Hansdorff space $(X, \tau)$ is closed.
\end{cor}
\begin{proof}
We show that $\forall x\in X \setminus K$ there is an open neighbourhood $U \supseteq X \setminus K$. Consequently $X \setminus K$ is open. Due to the Hansdorff property : $\forall y \in K$ there are disjoint open neighbourhoods $U_y, V_y$ with $x \in U_y$ and $y\in V_y$ where $\{V_y | y \in K\}$ is an open cover of K with finite subcover, say $\{V_i \}_{i \in \zeta}$. Then $U_i \coloneqq \cap_{i\in\zeta} U_i$ is an open neighbourhood of $x$ disjoint from $K$.
\end{proof}
\noindent Products of compact spaces are again compact in the product topology (Tychonoff's Theorem). The following implies that taking quotients preserves compactness as well (as the quotient map is continuous).

\begin{theorem}
If $f: X_1 \rightarrow X_2 $ is a continuous function between topological spaces $(X_i, \tau_i)$ and $K \subseteq X_1$ is compact, then $f(K)$ is compact. 
\end{theorem}
\begin{proof}
Let $\cup_{i \in I} \supseteq f(K)$ be an open cover. Since $V_i \coloneqq f^{-1}(U_i)$ is open, $\cup _{i \in I} V_i \supseteq K$ is an open cover containing a finite subcover $\{ V_i\}_{i \in \zeta}$. Hence, 
\begin{align*}
f(K) \subseteq f\left(\cup_{i\in\zeta} V_i\right) &= f\left(\cup_{i\in\zeta} f^{-1}(U_i)\right) \\
&= \cup_{i\in\zeta} f\left(f^{-1}(U_i)\right) \subseteq \cup_{i\in\zeta} U_i
\end{align*} 
\end{proof}

\noindent Since a subset of $\mathbb{R}$ is compact iff it is closed and bounded, we get: 
\begin{cor}
If $K$ is compact and $f: K \rightarrow \mathbb{R}$ continuous, then $f(K)$ has a minimum and maximum.
\end{cor}
\section{Metrizable Spaces}
\begin{itemize}
\item A \textbf{metric space} $(X, d)$ is a set $X$ with a distance function $d : X\times X \rightarrow [0, \infty)$ such that $\forall x, y, z \in X$
\begin{enumerate}[(i)]
\item $d(x, y) \geq 0$ with equality iff $x=y$.
\item $d(x, y) = d(y, x)$
\item $d(x, z) \leq d(x, y) + d(y, z)$ 
\end{enumerate} 
\item A subset $S$ of a metric space is \textbf{bounded} iff $\exists r\in[0, \infty) : B_{r}(x) \supseteq S$ for some $x$.
\item A metric space is called \textbf{complete} iff every Cauchy sequence converges
\item A topological space $(X, \tau)$ is called \textbf{metrizable} iff it is homeomorphic to a metric space (equivalently iff there is a metric such that $\{ B_{\epsilon(x)} \}_{\epsilon\geq 0, x \in X}$ is a base for $\tau$, i.e. the topology is \textbf{induced} by the metric) and \textbf{completely metrizable} iff it is homeomorphic to a complete metric space.
\end{itemize}

\begin{itemize}
\item In metric spaces: completeness $\Leftrightarrow$ sequential compactness
\item Every normal vector space becomes a metric space with $d(x, y) \coloneqq \Vert x - y\Vert$
\item Separable completely metrizable topological spaces are called \textbf{Polish spaces}
\item Since a subset of a metric space is closed iff it contains all limit points of a sequence, the topology of a metric space is completely determined by its converging sequences. In general, one has to consider so-called \textbf{nets}.
\end{itemize}

\begin{lemma}
Let $d_1, d_2$ be two metrics on $X$. They induce the same topology / are topologically equivalent if there are $k_1, k_2 > 0$ such that $\forall x, y \in X$, $k_1d_2(x, y) \leq d_1(x, y) \leq k_2 d_2 (x, y)$.
\end{lemma}

\begin{proof}
Define $B_{\epsilon}^{i}(x) \coloneq \{ y\in X | d_i(x, y) < \epsilon\}$. We have to show that $B_{\epsilon}^{1}(x)$ is open in the topology induced by $d_2$. By symmetry, the same holds for $1\leftrightarrow 2$. $\forall y \in B_{\epsilon}^{1}(x) \exists \delta > 0 : B_{\delta}^{1}(y) \subseteq B_{\epsilon}^{1}(x)$. Hence, $B_{\delta/k_1}^{2}(y) \subseteq B_{\epsilon}^{1}(x)$. So for any $y\in B_{\epsilon}^{1}(x)$ there is an open $B^{2}$ neighbourhood inside $B^{1}$, which implies that $B^{1}$ is open with respect to $\tau_2$.
\end{proof}

\begin{definition}
\textbf{Isometries}. Let $(X_i, d_i)_{i \in \{1,2\}}$ be a metric space.
\begin{itemize}
\item $f: X_1 \rightarrow X_2$ is called \textbf{isometry} iff $\forall x,y\in X_1: d_1(x, y) = d_2(f(x), f(y))$.
\item The two metric spaces are called \textbf{isometric} iff there is a bijective isometry $f: X_1 \rightarrow X_2$.
\end{itemize}
\end{definition}
\noindent Note that an isometry is automatically injective and that the inverse of a bijective isometry is again an isometry.
\begin{definition} 
\textbf{Completion}. A complete metric space $(y, delta)$ is a \textbf{completion} of the metric space $(X, d)$ iff there exists an isometry $f: X\rightarrow Y$ such that $\overline{f(X)} = Y $.
\end{definition}
\noindent In practice, one often identifies $X$ with $f(X)$ and thus considers $X$ as a dense subset of $Y$.

\begin{theorem}
\textbf{Existence of a completion} Every metric space $(X, d)$ has a completion.
\end{theorem}
\begin{proof}
(Sketch) Define $y\coloneqq \{ Cauchy-sequence in X\} \setminus \sim $ where $(x_n) \sim (y_n) \Leftrightarrow \lim_{n \rightarrow \infty} d(x_n, y_n) = 0$ and $\delta Y\times Y \rightarrow [0, \infty)$.  $\delta([x], [y]) \coloneqq \lim_{n\rightarrow\infty} d(x_n, y_n)$. $(Y, \delta)$ can be shown to be a complete metric space. Take $f: X\rightarrow Y$ such that $f(a) \coloneqq (a, a, a, ...)$. This is an isometry, since $\delta(f(a), f(b)) = \lim_{n\rightarrow\infty} d(a, b) = d(a, b)$. 
\\ \\
Moreover, $\forall [x] \in Y$, we can construct a sequence in $f(x)$ that convergest to $[x]$. In fact, for $y_n \coloneqq f(x_n)$ we obtain $\delta(y_n, [x]) = \lim_{n\rightarrow\infty} d(y_{n,m}, x_m) = \lim_{n\rightarrow\infty} d(x_n, x_m)$. So $\delta(y_n, [x]) \rightarrow 0$ for some $n \rightarrow \infty$ since $(x_n)$ is a Cauchy sequence. 
\end{proof}
\begin{itemize}
\item The completion of a metric space is unique in the sense that any two completions are isometric. 
\item Since $f(x)$ is dense in $Y$, the completion of a separable metric space is again separable. 
\item If there is a scalar product or norm, the completion preserves this  structure.
\end{itemize}

\begin{example}
From the homework we know that $(C([0, 1], \mathbb{K}, \Vert\cdot\Vert_{\infty})$ with $\mathbb{K} \in \{ \mathbb{R}, \mathbb{C}\}$ is complete.
\end{example}

\begin{theorem}
\textbf{Weierstrass Approximation Theorem}. The set $X$ of polynomials on $[0, 1]$ is dense in $Y$ (with respect to the metric induced by $\Vert \cdot \Vert_{\infty}$).
\end{theorem}
\noindent Consequently, $Y$ is the completion of $X$ with isometry $f: X \rightarrow Y, f \mapsto f$.

\begin{theorem}
If $S \subseteq X$ is a subset of a complete metric space $(X, d)$, then $(S, d)$ is compelte iff $S$ is closed.
\end{theorem}
\begin{proof}
$(\Rightarrow)$: Let $(X_n)$ be a sequence in $S$ that converges to $x\in X$. $(X_n)$ is thus a Cauchy sequence and $x\in S$ by completeness of $S$. \\ \\
$(\Leftarrow)$: Let $(X_n)$ be a Cauchy sequence in $S$. By completeness of $X$ it converges to some $x \in X$. Since $S$ is closed, $x \in S$. 
\end{proof}
\begin{itemize}
\item The $(\Rightarrow)$ part actually shows that every complete subspace of a (not necessarily complete) metric space is closed.
\item The theorem enables us to prove \textbf{completeness of a space} by 
\begin{enumerate}[(i)]
\item Considering it as a subspace of a larger complete space 
\item Proving closedness
\end{enumerate}
\end{itemize} 

\begin{theorem}
\textbf{Baire's Category Theorem} If $(U_n)$ is a sequence of open, dense subsets of a complete metric space $(X, d)$, then the intersection $U \coloneqq \cup_{n in \mathbb{N}}$ is dense in $X$
\end{theorem}
\begin{proof}
We want to show that $B_{\epsilon_0}(x_0) \cup U \neq \varnothing$ for any $\epsilon_0 > 0, x_0 \in X$. Define $\overline{B}_{\epsilon}(x) \coloneqq \{ y\in X|d(x,y)\leq \epsilon\}$ and note that $\overline{B}_{\epsilon} \subseteq B_{2\epsilon}(x)$. Since $U_n$ is open and dense, $U_n \cup B_{\epsilon_0}(x_0)$ is open and non-empty. Consequently, there is $x_1\in X$ and $\epsilon_1 > 0$ such that $U_1 \cup B_{\epsilon_0}(x_0) \supseteq B_{2\epsilon_0}(x_0) \supseteq \overline{B}_{\epsilon_1}(x_1)$. Now construct a sequence of points $(x_n)$ in $X$ and radii $(\epsilon_n)$ such that $\overline{B}_{\epsilon_n}(x_n) \subseteq U_n \cup B_{\epsilon_{n-1}}(x_{n-1})$ and $\epsilon_n \in (0, 2^{-n}\epsilon_0]$. By construction, $\overline{B}_{\epsilon_n}(x_n) \subseteq \overline{B}_{n-1}(x_{n-1})$ and $\epsilon_n \rightarrow 0$ for $n\rightarrow \infty$. In particular, $d(x_n, x_m)\leq \epsilon_N = 2^{-N}\epsilon_0, \forall n,m >N$. So, $(x_n)$ is a Cauchy sequence and by completeness $\lim_{n\rightarrow 0} x_n = x_{\infty} \in \cup_{n\in \mathbb{N}} \overline{B}_{\epsilon_n}(x_n)$. Since $\forall n \in \mathbb{N}: x_{\infty} \in \overline{B}_{\epsilon_n}(x_n)\subseteq U_n$ we have $x_{\infty} \in B_{\epsilon_0}(x_0) \cup U$.
\end{proof}

A useful consequence is formulated in terms of \textbf{nowhere dense}.
\begin{definition}
A subset $S$ of a topological space is called \textbf{nowhere dense} iff its closure has empty interior : $\text{int}(\overline{S} = \varnothing$.
\end{definition}
Note that $S$ is nowhere dense iff its closure is. 

\begin{cor}
A complete metric space is never the countable union of nowhere dense sets
\end{cor}
\begin{proof}
Assume $X = \bigcup_{n\in\mathbb{N}} c_n$ where $c_n$ is closed, that is, $U_n \coloneqq X \setminus c_n$ open. Then $\bigcup_{n\in\mathbb{N}} U_n = \bigcup_{n}(X\setminus c_n) = X \setminus \bigcup_{n} c_n = \varnothing$. By Baire's category theorem, at least one of the $U_n$is not dense. Then, however, $\text{int}(c_n) = X \setminus \overline{U_n} \neq \varnothing$.
\end{proof}

\section{Normed and Banach Spaces}

\begin{definition}
Let $X$ be a $\mathbb{K}$-vector space, $\mathbb{K}\in\{\mathbb{R}, \mathbb{C}\}$.
\begin{itemize}
\item $\Vert\cdot\Vert : X \mapsto [0, \infty)$ is called a \textbf{seminorm} iff $\forall x, y \in X, \forall \alpha\in\mathbb{K}$
\begin{enumerate}[(i)]
\item $\Vert x + y \Vert \leq \Vert x \Vert + \Vert y \Vert$
\item $\Vert\alpha x \Vert = |\alpha| \Vert x \Vert$
\end{enumerate}
\item A seminorm is a \textbf{norm} iff $\norm{x} = 0 \Rightarrow x = 0$. $(X, \norm{\cdot})$ is then called a \textbf{normed space}.
\item A \textbf{Banach Space} is a normed space that is complete
\item The \textbf{linear span} of a subset $S \subseteq X$ is defined as $\text{span}(S) \coloneqq \left\{ \sum_{i = 1}^{n} \alpha_i x_i | n\in \mathbb{N} x_i \in S, \alpha_i \in \mathbb{R}\right\}$
\end{itemize}
\end{definition}
\begin{itemize}
\item \textbf{Completeness} refers to the metric space with $d(x, y) \coloneqq \norm{x - y}$.
\item A metric is induced by a norm iff $\forall x,y,z \in X \forall\alpha\in\mathbb{K}:$
\begin{enumerate}[(i)]
\item $d(x+z, y+z) = d(x, y)$
\item $d(\alpha x, \alpha y) = |\alpha| d(x, y)$
\end{enumerate}
To see that this is sufficient for the existence of a norm, consider $\norm{x} = d(x , 0)$ and verify the necessary properties. 
\item Examples of \textbf{normed sequence spaces}
\begin{enumerate}[(i)]
\item $l_{\infty} \coloneqq \left\{ x \in \mathbb{K}^{\mathbb{N}} \mid \norm{x}_{\infty} \coloneqq \sup_{n} |x_n| \leq \infty \right\}$
\item $c_0 \coloneqq \left\{ x \in \mathbb{K}^{\mathbb{N}} \mid \lim_{n \rightarrow \infty} x_n = 0 \right\}$ with $\norm{\cdot}_{\infty}$
\item $l_{p} \coloneqq \left\{ x \in \mathbb{K}^{\mathbb{N}} \mid \norm{x}_p \coloneqq \left( \sum_n |x_n| \right)^{1/p} <\infty \right\}, p\in[1, \infty)$
\item $c_c \coloneqq \left\{ x \in \mathbb{K}^{\mathbb{N}} \mid \left| \{n \mid x_n \neq 0\}\right| < \infty \right\} $
\end{enumerate}
As sets, these satisfy $c_c \subseteq l_p \subseteq l_q \subseteq c_0 \subseteq l_0$ for $p \leq q < \infty$. $\left( l_{\infty}, \norm{\cdot}_{\infty} \right)$ is a Banach space. Since $c_0$ is a closed subset in $l_\infty$, $\left(c_0, \norm{\cdot}_{\infty}\right)$ is also a Banach space. Also $\left(l_p, \norm{\cdot}_{p}\right)$ are Banach spaces. $c_c$ is dense in $l_p$ for $p < \infty$ with respect to $\norm{\cdot}_p$ and $c_c$ is dense in $c_0$ with respect to $\norm{\cdot}_{\infty}$. Since $c_c \neq l_p$, it is not a Banach space for any $\norm{\cdot}_p$, $p\in[1, \infty]$.
\item $l_\infty$ can be generalized to arbitrary sets $X$: 
\begin{align*}
l_\infty(X) \coloneqq \left\{ f \in \mathbb{K}^{\mathbb{N}} \mid \norm{f}_\infty \coloneqq \sup_{x \in X} |f(x)| < \infty\right\}.
\end{align*}
$\left(l_\infty(x), \norm{\cdot}_\infty\right)$  then turns out to be a Banach space.
\item Examples of \textbf{normed function spaces}:
\begin{enumerate}[(1)]
\item \textbf{Continuous functions}: for any topological space $(X, \tau)$ define,
\begin{enumerate}[(i)]
\item $c_b(x) \coloneqq \left\{ f : X \mapsto \mathbb{K} \mid f \text{ is continuous and } \norm{f}_{\infty} \coloneqq \sup_{x \in X} |f(x)| < \infty \right\}$
\item 
\end{enumerate}

\end{enumerate}
\end{itemize}


\end{document}
